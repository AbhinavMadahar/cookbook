\documentclass{amsbook}

\newthm{rec}{Recipe}
\newthm{cor}{Corollary}

\author{Abhinav Madahar}
\affil{Rutgers University}
\email{abhinav.madahar@rutgers.edu}
\date{\today{}}

\begin{document}
\maketitle

\tableofcontents{}

\chapter{All Ingredients and Tools}
Every recipe in this book can be made using the following 16 ingredients:
\begin{itemize}
	\item Tofu
	\item Instant ramen
	\item Sesame oil
	\item Soy sauce
	\item Lemon pepper
\end{itemize}

Additionally, you can change the recipes by using the following 16 ingredients:
\begin{itemize}
	\item Plantains
\end{itemize}

\chapter{Recipes}
\begin{rec} Tofu with ramen. \end{rec}

\textit{Ingredients:}

\begin{itemize}
	\item Extra firm tofu, half-block
	\item Instant ramen, 1 block
	\item Sesame oil, 1 tbsp
	\item Soy sauce, 1 tbsp
	\item Lemon pepper, 1 tsp
\end{itemize}

\textit{Required tools.}
\begin{itemize}
	\item Pan
	\item Spatula
	\item Filter
\end{itemize}

\textit{Directions.}

\begin{itemize}
	\item Begin enough water to submerge ramen block.
	\item Chop tofu into 1\ cm\textsuperscript{2}-sized blocks.
	\item Warm tofu and plantains in a pan with medium-low heat while applying
		the sesame oil.
	\item Once the water is boiling, put the ramen in.
	\item Once the ramen is breaking up into individual noodles, pour out of
		water over the filter.
	\item Pour the ramen into the same pan as the tofu.
	\item Add soy sauce and lemon pepper.
\end{itemize}

QED.

\begin{cor} Tofu with ramen and plantains. \end{cor}
\textit{Ingredients.} the same as before, but add a medium or large ripe
(yellow and brown) plantain.

\textit{Directions.} the same as before, but chop the plantain into slices 1\
cm thick and cook with tofu.

\end{document}
